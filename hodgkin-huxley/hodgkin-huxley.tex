\documentclass[12pt]{article}
\usepackage[a4paper]{geometry}

\usepackage[lastexercise]{exercise}

%%%%%%%%%%% Preamble for exercise sheets
\pagestyle{empty}
\usepackage{vmargin}
\setmarginsrb{2.5cm}{2.5cm}{2.5cm}{2.5cm}{0cm}{0cm}{0cm}{0cm}
\setlength{\parindent}{0cm}
\linespread{1.0}
\newcommand{\Title}[1]{\makebox[\textwidth][c]{{\large\scshape{#1}}}\\[5mm]}
\newcommand{\What}[1]{\makebox[\textwidth][c]{\large{#1}}\\[5mm]}
\newcommand{\Who}[1]{{\parbox[t]{\linewidth}{\centering\large{#1}}\\[1.0cm]}}
%%%%%%%%%%%

\begin{document}

\What{Neuronal Dynamics: Python Exercises}
\Who{Professor Wulfram Gerstner\\Laboratory of Computational Neuroscience, EPF Lausanne}
\Title{Hodgkin-Huxley (HH) Neuron Model}

\begin{Exercise}[title=Numerical Integration of HH model of the Squid Axon]

Download HH.py from the book's webpage.  HH.py is a python module containing 4 main functions:  HH\_Step, HH\_Ramp, HH\_Sinus and HH\_ForwardEuler.  The later is a subroutine used by the first 3 to perform the numerical integration.  With those, you can simulate a step current, a ramp current or a sinusoidal current injected in the squid axon.  The specific formulas implemented are described on pages 32 and 33 of the book.  Once you have started ipython -pylab in the directory containing HH.py, simply type:
\begin{verbatim}
>> import HH
\end{verbatim}
to port HH.py onto your current session.  Then you can simulate a step current in a Hodgkin-Huxley model by typing:
\begin{verbatim}
>> HH.HH_Step()
\end{verbatim}
which should trigger a plot with three panels.  To have information on the arguments of the function, simply type:
\begin{verbatim}
>> HH.HH_Step?
\end{verbatim}
or open HH.py in any text editor.\\

\Question{What is the lowest step current amplitude for generating at least one spike? Hint: use binary search on $I_{amp}$, with a 0.1 $\mu A$ resolution.}
\Question{What is the lowest step current amplitude to generate repetitive firing?}
\Question{What is the minimum current required to make a spike when the current is slowly increased (ramp current waveform) instead of being increased suddenly?}
\Question{What is the current threshold for repetitive spiking if the density of sodium channels is increased by a factor of 1.5? (You need to change the maximum conductance of sodium channel.)\\

Hint:  You can change the parameters of the model in the appropriate section of HH.py; use any text editor to save the change.  To actualize the change you have saved, you must type in your current ipython workspace:
\begin{verbatim}
>> reload(HH)
\end{verbatim}
}

\Question Look at HH\_Step( I\_amp = -5) and HH\_Step(I\_amp = -1).  What is happening here?  To which gating variable do you attribute this rebound spike?

\end{Exercise}
\end{document}