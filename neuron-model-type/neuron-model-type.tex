\documentclass[12pt]{article}
\usepackage[a4paper]{geometry}

\usepackage[lastexercise]{exercise}

%%%%%%%%%%% Preamble for exercise sheets
\pagestyle{empty}
\usepackage{vmargin}
\setmarginsrb{2.5cm}{2.5cm}{2.5cm}{2.5cm}{0cm}{0cm}{0cm}{0cm}
\setlength{\parindent}{0cm}
\setlength{\parskip}{2mm}
\linespread{1.0}
\newcommand{\Title}[1]{\makebox[\textwidth][c]{{\large\scshape{#1}}}\\[5mm]}
\newcommand{\What}[1]{\makebox[\textwidth][c]{\large{#1}}\\[5mm]}
\newcommand{\Who}[1]{{\parbox[t]{\linewidth}{\centering\large{#1}}\\[1.0cm]}}
%%%%%%%%%%%

\begin{document}

\What{Neuronal Dynamics: Python Exercises}
\Who{Professor Wulfram Gerstner\\Laboratory of Computational Neuroscience, EPF Lausanne}
\Title{Neuron Model: Type I or Type II?}

\begin{Exercise}[]

For this computer exercise, download TypeX.py, TypeY.py, tools.py, fvsI.py
from the book's webpage.  One of TypeX.py or TypeY.py is
a FitzHugh-Nagumo Type II model, and the other is a Morris-Lecar Type
I model.  The objective of this exercise is to find out which is which.

Start pylab in the directory containing the
downloaded files.  You should then be able to import the two models:

\begin{verbatim}
>> import TypeX
>> import TypeY
\end{verbatim}

\Question What is the threshold current for repetitive firing for
TypeX, TypeY?

To this end, use the TypeX.PlotStep or TypeY.PlotStep to plot the response to
a step current which starts after 100ms (to let the system
equilibrate) and lasting 1000ms (to detect repetitive firing with a
long period):

For example using:

\verb^>> TypeY.PlotStep(I_amp=0.5,Step_tstart=100,Step_tend=1000,tend=1000)^

Exploring various values of \verb^I_amp^, find the range in which the threshold
occurs, to a precision of 0.01.

Already from the voltage response near threshold you might have an idea
which is type I or II, but let's investigate further ...

\Question Plot on one axis the response to short pulses near
threshold, and interpret the results.  Which is Type I, II ?

Example:

\begin{verbatim}
>> figure()
>> t,v,w,I = TypeY.Step(I_amp=1.05,Step_tstart=100,Step_tend=110,tend=300)
>> plot(t,v)
>> t,v,w,I = TypeY.Step(I_amp=1.1,Step_tstart=100,Step_tend=110,tend=300)
>> plot (t,v)
>> t,v,w,I = TypeY.Step(I_amp=1.15,Step_tstart=100,Step_tend=110,tend=300)
>> plot (t,v)
\end{verbatim}

\newpage

\Question Plot f-vs-I curves for each TypeX, TypeY

Provided in tools.py is a function to determine the spike times from
t,v:

\begin{verbatim}
>> import tools
>> t,v,w,I = TypeX.Step(I_amp=0.40,Step_tstart=100,Step_tend=1000,tend=1000)

>> st = tools.spiketimes(t,v)
>> print st
[ 102.9  146.1   189.1 ... ]
\end{verbatim}

Using this function one can calculate an estimate of the firing rate:

First calculate the inter-spike intervals (time difference between
spikes) using this elegant indexing idiom:

\verb^>> isi = st[1:]-st[:-1]^

Then find the mean and take the reciprocal (pay attention when converting from 1/ms to
Hz) to yield the firing-rate:

\begin{verbatim}
>> f = 1000.0/mean(isi)
\end{verbatim}

This function is provided in \verb^tools.py^ as the function \verb^tools.f^

Now let's use it to plot an f-vs-I curve for each TypeX/TypeY

The script to plot an f-vs-I curve is
provided on the moodle as fvsI.py. You can execute it by using:

\begin{verbatim}
>> execfile('fvsI.py')
\end{verbatim}

\Question In the file fvsI.py change TypeX to TypeY, and change the \verb^I^ range to zoom in near the threshold. Then execute it again.

Which is Type I and which is Type II?

\end{Exercise}

\end{document}